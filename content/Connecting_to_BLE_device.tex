\chapter{Connecting to aMussel over Bluetooth}
\label{ch:connecting_to_ble}

\begin{figure}[htb]
    \centering
	  \includegraphics[width=0.5\linewidth]{figures/USB_Dongle.png}
	\caption{BLE USB dongle (CY5670)}
	\label{fig:ble_hardware}
\end{figure}

Necessary hardware components:
\begin{itemize}
	\item Bluetooth (BLE) module on MainBoard (MB) -- each aMussel comes with an embedded BLE module
	\item BLE dongle that needs to be connected to user's PC USB port -- \href{http://www.cypress.com/documentation/development-kitsboards/cy5670-cysmart-usb-dongle}{dongle used for subCULTron} \\
	The purpose of BLE dongle is to provide UART bridge between MB and PC. It's main purpose is to establish basic serial communication between MB on aMussel and PC.
\end{itemize}

In order to connect to aMussel over Bluetooth, the following prerequisites have to be met:
\begin{enumerate}
	\item BLE dongle has to be programmed.
	\item BLE module on MB has to be programmed with a certain aMussel ID.
\end{enumerate}

\section{Programming BLE dongle}

The code for BLE dongle is provided in subCULTron's GitHub page (\href{https://github.com/subCULTron-project}{link}), \texttt{BLE-Dongle} repository. To program BLE dongle you need to install PSoC Creator 4.0 IDE (\href{http://www.cypress.com/documentation/software-and-drivers/psoc-creator-software-archive}{here}) and checkout the needed git repository.

Then:
\begin{enumerate}
	\item Open the \texttt{BLE-Dongle} project in PSoC Creator. Build the project (right click on project name > Build \texttt{BLE\_DONGLE}).
	\item Set \texttt{BLE\_DONGLE} project to active (right click on project name > Set As Active Project).
	\item Connect BLE dongle to the PC's USB port 
	\item Program the dongle (\textit{program} button in PSoC creator or Ctrl+F5).
	\item If prompted, select a device to program.
\end{enumerate}

\section{Programming BLE module}

BLE module can be programmed using MiniProg programmer. The code is provided in \texttt{BLE\_UART\_BRIDGE} project of PSoC creator \texttt{MU31S-000} workspace, which can be found on project's GitHub page, \texttt{MainBoard-middleware} repository.

\begin{figure}[htb]
    \centering
	  \includegraphics[width=\linewidth]{figures/BLE_device_name.PNG}
	\caption{Setting the BLE device name}
	\label{fig:ble_device_name}
\end{figure}

Steps:
\begin{enumerate}
	\item Modify the project to define aMussel ID. To do so, open TopDesign.cysch and double click on BLE module. Go to GAP Settings > General and enter aMussel ID in Device Name field (see Figure \ref{fig:ble_device_name}). Naming convention is as follows: \texttt{'aMXXX'}, where \texttt{'XXX'} represents aMussel ID in range $[000-999]$, e.g. \texttt{'aM001', 'aM023'} \ldots
	\item Build the project (right click on project name > Build \texttt{BLE\_UART\_BRIDGE}).
	\item Set \texttt{BLE\_UART\_BRIDGE} project to active (right click on project name > Set As Active Project).
	\item Connect MiniProg to BLE module and the PC.
	\item Program the device (\textit{program} button in PSoC creator or Ctrl+F5).
	\item If prompted, select a device to program.
\end{enumerate}

\section{Communicating with MU using BLE}

\begin{figure}[htb]
    \centering
	  \includegraphics[width=\linewidth]{figures/Docklight_COM.PNG}
	\caption{Selecting the COM port}
	\label{fig:docklight_com}
\end{figure}

\begin{figure}[htb]
    \centering
	  \includegraphics[width=\linewidth]{figures/Docklight_BLE_dongle.PNG}
	\caption{BLE bridge status}
	\label{fig:docklight_ble_bridge_status}
\end{figure}

\begin{figure}[htb]
    \centering
	  \includegraphics[width=0.6\linewidth]{figures/USB_dongle_parts.png}
	\caption{CY5670 USB Dongle}
	\label{fig:usb_dongle_parts}
\end{figure}

\begin{figure}[htb]
    \centering
	  \includegraphics[width=\linewidth]{figures/Docklight_aMussel_selection.PNG}
	\caption{Selecting a device to connect to}
	\label{fig:docklight_aMussel_selection}
\end{figure}

To successfully send and receive data between aMussel and PC, we recommend the following setup:
\begin{enumerate}
	\item Connect the programmed BLE dongle to your PC.
	\item Open Docklight \footnote{\url{docklight.de/}} (or any other tool for serial communication).
	\item Connect to the correct COM port. Port can be selected by double clicking on top right sections of status bar (indicated with red rectangle in Figure \ref{fig:docklight_com}). After selecting the port, make sure the settings are matching the ones in the figure. Now you can simply press the play button to establish connection.
	\item When connected to BLE dongle, you should see the output in your serial communication program as presented in Figure \ref{fig:docklight_ble_bridge_status}. If the output is not there or the program is stuck for any reason, you can always reset it by pressing the user button on dongle (see Figure \ref{fig:usb_dongle_parts}).
	\item The next step is to select device ID you wish to connect to. You do so by sending a sequence representing device's unique identifier. For example, if BLE board on aMussel is programmed to have ID \texttt{001}, the sequence needs to match it. For reference, look at Figure \ref{fig:docklight_aMussel_selection}.
	\item Upon connection establishment the program prints out: \\
	\hspace{10pt}\parbox[t]{\linewidth}{\texttt{Server with matching custom service discovered... \\
		Connection established \\
		Notifications enabled \\
		Start entering data:
	}}
	\item Now the connection is established and you can start sending and receiving data.
\end{enumerate}
